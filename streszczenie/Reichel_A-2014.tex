%& --translate-file=cp1250pl
\documentclass[10pt,a5paper]{book}

\usepackage{polski}



\tolerance=5000
\clubpenalty=100000
\widowpenalty=10000

%praca nowa d�ugo��
\textheight=170mm
\textwidth=118mm
\topmargin=-20mm
\oddsidemargin=-10mm
\evensidemargin=-10mm

\raggedbottom

\usepackage{graphicx}

\usepackage{indentfirst}
\usepackage{sectsty}

\sectionfont{\centering\large\normalfont}

\subsectionfont{\centering\normalsize\normalfont}



\usepackage{fancyhdr}
\pagestyle{fancy}



\lhead{\begin{minipage}[b]{5.2cm} %11.8cm}
\footnotesize IV Konferencja Zastosowa� Matematyki\\[-0.4ex]
\footnotesize w~Technice, Informatyce i~Ekonomii\rule[-1ex]{1ex}{0ex}
\end{minipage}}
\chead{}
\rhead{\begin{minipage}[b]{2.8cm}
\footnotesize {\ }\\
\footnotesize Gliwice, 18.09.2014~r.
\end{minipage}}
\lfoot{}
\cfoot{\footnotesize\thepage}
\rfoot{}
\renewcommand{\headrulewidth}{0.4pt}
\addtolength{\headheight}{12pt}



\begin{document}



%%%%%%%%%%%%%%%%%%%%%%%%%%%%%%%%%%%%%%%%%%%%%%%%%%%%%%%%%%%%%%%%%%%%%%%%%%%%%%%%%%%%%%%%%%%%%%%%%%%%%%%%%%%%%%%%%%%%%%%%%%

%\pagebreak

\vspace*{-1ex}

\begin{center}
\large\textbf{Por�wnanie wybranych algorytm�w optymalizacyjnych z~modelem probabilistycznym}
\end{center}
\bigskip

\centerline{\large{Anna Reichel$^1$, Iwona Nowak$^2$}}\label{reichel}
\medskip

\begin{center}
\footnotesize{ $^{1}$Wydzia�� Matematyki Stosowanej, Politechnika �l�ska\\
ul. Kaszubska 23, 44-100 Gliwice\\
$^2$Instytut Matematyki, Politechnika �l�ska\\
ul. Kaszubska 23, 44-100 Gliwice\\
$^2$\texttt{Iwona.Nowak@polsl.pl}\\
}
\end{center}
\vspace*{2ex}


%%%%%%%%%%%%%%%%%%%%%%%%%%%%%%%%%%%%%%%%%%%%%%%%%%%%%%%%%%%%%%%%%%%%%%%%%%%%%%%%%%%%%%%%%%%%%%%%%%%%%%%%%%%%%%%%%%%%%%%%%

\section*{Streszczenie}

W coraz liczniejszej grupie algorytm�w populacyjnych, algorytmy wykorzystuj�ce modele probabilistyczne zaczynaj� w ostatnim czasie
odgrywa� coraz wi�ksz� rol�.
S� to najcz�ciej metody o strukturze bardzo podobnej do struktury algorytmu ewolucyjnego, 
z~t� r�nic�, �e kolejne pokolenia osobnik�w/rozwi�za� generuje si� na bazie 
modelu probabilistycznego populacji rozwi�za� obiecuj�cych,
nie za� jako efekt krzy�owania b�d� mutacji osobnik�w z populacji bie��cej.

W pracy zaprezentowano dwie wersje algorytm�w z modelem populacyjnym: algorytm PBIL (ang. Population-Based Incremental Learning)
oraz cGA (ang. Compact Genetic Algorithm) [1,2,3]. Dzia�anie obu metod por�wnano na wybranych funkcjach testowych.


%%%%%%%%%%%%%%%%%%%%%%%%%%%%%%%%%%%%%%%%%%%%%%%%%%%%%%%%%%%%%%%%%%%%%%%%%%%%%%%%%%%%%%%%%%%%%%%%%%%%%%%%%%%%%%%%%%%%%%%%%

\subsection*{Bibliografia}

\begin{list}{}{\itemindent=0pt
\labelwidth=15pt
\leftmargin=20pt
\labelsep=5pt}

\item[1.]
G.R. Harik, F.G. Lobo, D.E. Goldberg,
\textit{The compact genetic algorithm},
IEEE Transactions on Evolutionary Computation \textbf{3}, no.~4 (1999), 287--297.

\item[2.]
F.O.~Karray, C.~de Silva, 
\textit{Soft Computing and Intelligent Systems Design},
Addison Wesley, New York 2004.

\item[3.]
M.~Pelikan,
\textit{Bayesian Optimization Algoritm: from single level to hierarchy},
PhD Thesis, University of Illinois at Urbana-Champaign, 2002.

\end{list}

%%%%%%%%%%%%%%%%%%%%%%%%%%%%%%%%%%%%%%%%%%%%%%%%%%%%%%%%%%%%%%%%%%%%%%%%%%%%%%%%%%%%%%%%%%%%%%%%%%%%%%%%%%%%%%%%%%%%%%%%%


%%%%%%%%%%%%%%%%%%%%%%%%%%%%%%%%%%%%%%%%%%%%%%%%%%%%%%%%%%%%%%%%%%%%%%%%%%%%%%%%%%%%%%%%%%%%%%%%%%%%%%%%%%%%%%%%%%%%%%%%%

\end{document}

